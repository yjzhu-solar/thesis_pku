% Copyright (c) 2014,2016 Casper Ti. Vector
% Public domain.

\specialchap{序言}
太阳上存在着一系列的爆发活动,不同的爆发活动所在的大气高度不同,释放的能量也能相差数个量级。一般来说,大尺度的爆发活动能够引起太阳及行星际空间大规模的磁场扰动,并伴随着大量等离子体物质进入行星际空间。这些携带磁场的等离子体物质和另外一些加速的高能粒子能够对地球和一系列恒星的空间环境产生重要影响。而伴随着人类观测手段的发展,大量小尺度的爆发活动也在不同波段被观测到。尽管单个释放的能量较小,但广泛的分布使它们有可能为加热色球以及日冕大气提供了重要贡献。耀斑这类爆发事件在恒星上也广泛存在,对它们的探测同样起步于上世纪七八十年代。近年来,一些M矮星上巨大的恒星耀斑也被地面和空间望远镜所观测到。相比于一般的太阳耀斑,它们释放的能量更多,并能在可见光连续谱上产生重要增强。由于M矮星质量较小,数量众多(根据初始质量函数),寿命更长,附近的行星更容易被视向速度和掩星法等一系列手段探测到。但这些系外行星的可宜居性一直受到这些剧烈恒星活动,包括恒星耀斑和恒星CME的挑战。

尽管这些爆发活动大小各异,对应的磁场结构也千差万别,但它们普遍都能加热色球及过渡区大气,产生一些可见光及紫外波段的谱线增强,如Mg \textsc{ii},H$\alpha$, C \textsc{ii}等。这些谱线的轮廓特征变化为我们理解爆发活动中的大气演化提供了重要的诊断依据。观测上一个重要的问题是这些谱线普遍是光学厚的,即大量在谱线轮廓内逃逸的光子形成于散射过程中,一般的光学薄反演不再适用。需要从数值模拟出发,计算辐射转移过程来理解这些谱线对应的大气演化结构。当前的一维辐射转移已经揭示了一些谱线在爆发活动中的谱线特征,但深入理解这些谱线的形成和演化仍然需要更多的研究和更加强大能同时描述色球、过渡区和日冕的三维辐射磁流体力学代码。当前研究人员也在利用一些新技术,如机器学习等方式来利用当前的辐射流体力学模拟结果反演大气参数。相信随着研究的不断深入,这些谱线诊断能够为我们描绘一个完整的耀斑图像提供重要的帮助。

这篇毕业论文主要包含了以下几个部分,在第\ref{chap:1}章中,我将对目前太阳和M矮星上观测到的大规模和小规模的爆发活动以及观测到的紫外谱线特征做一个简单介绍。为了方便读者理解我们的辐射转移计算过程,这部分还介绍了一些与辐射转移有关的物理概念,如局部热动平衡等。在第\ref{chap:2}章中,我将会较为具体地介绍我们使用的相关辐射流体力学代码RADYN和辐射转移代码RH。在第\ref{chap:3}章中,我将会讨论一个由RHESSI观测数据驱动的耀斑一维辐射流体力学模拟以及计算得到的合成光谱与实际观测的比较,并对这些谱线在耀斑色球和过渡区大气中的形成条件以及辐射流体力学模拟本身的不足进行讨论。在第\ref{chap:4}章中,我将利用一个针对M矮星耀斑的不同电子束参数的模拟结果计算其中Mg \textsc{ii}谱线的轮廓和演化,并和仅有的一少部分观测做一个简单比较。在第\ref{chap:5}章中,我将通过一些半经验的方法在低层大气中人为提高温度,并计算在其在Mg \textsc{ii}等谱线轮廓上产生的响应,同时与观测到的紫外爆发及Ellerman炸弹的光谱进行对比。每章中都会对取得的结果以及遇到的问题进行一定程度的讨论,因此我们在最后仅作出一些归纳性的总结和对未来工作的展望。

总的来说,这篇毕业论文是尽可能在较短的时间内尝试对多个辐射流体力学模拟可以研究的对象做一个走马观花式的、浅尝辄止的探索。我们发现了一些有趣的结果,但是由于作者本身的不成熟以及时间有限也遇到了不少问题。我个人更愿意把这篇论文看作是一个投石问路的角色,希望能够对我之后相关的研究内容有一定的吸取教训和参考借鉴意义。

由于本人才疏学浅,毕业论文也是草草赶制而成,如有错误请不吝赐教。

\begin{flushright}
	\emph{朱英杰} 
	
	\emph{2019年6月于燕园}
\end{flushright}



% vim:ts=4:sw=4
