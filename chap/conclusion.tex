% Copyright (c) 2014,2016 Casper Ti. Vector
% Public domain.

\chapter{结论}\label{chap:con}
本文通过一维辐射流体力学代码RADYN和一维辐射转移代码RH进行了一系列太阳和M矮星上爆发活动的模拟,包含由观测数值驱动的太阳上的X级耀斑的5F11非热电子加热模拟,对M矮星上发生的白光耀斑的不同非热电子束加热的参数研究,以及对太阳上的低高度和小尺度的爆发活动的半经验的模拟。尽管有些模拟过程非常的简单粗暴和直接,但是我们仍然从这些数值模拟当中发现了一些有趣的结论:
\begin{enumerate}
	\item 我们在RH代码中引入了新的Stark致宽计算方法,并发现对于Mg \textsc{ii}谱线仍需要在Stark宽度上乘上一个30倍的因子才能较好的再现Mg \textsc{ii} h,k和三重线在耀斑带上远线翼出现剧烈辐射增强的谱线轮廓。对于C \textsc{ii}远线翼的辐射,我们同样可以通过人为增大其Stark致宽效应来在远线翼上产生和观测拟合较好的谱线轮廓。这些因子可能来源于非热电子对Stark致宽的贡献和对大气电子密度的错误计算等原因。
	\item 我们首次在自洽的大气中计算得到了单发射峰的非线心反转的Mg \textsc{ii} h和k线,它们的线心形成于高度压缩的过渡区下电子密度约为$8\times 10^{14}\ \mathrm{cm^{-3}}$的区域内。如此高的电子密度保证了线心形成范围内的近似局部热动平衡,使谱线不再出现线心反转的轮廓。
	\item 我们发现Mg \textsc{ii}三重线的形成高度在模拟耀斑衰减相时线心形成高度和Mg \textsc{ii} h和k线非常相近,因此它们在耀斑过程中可能不再是低色球加热的良好定量诊断。
	\item 我们的5F11模拟能够较好地再现Mg \textsc{ii} h和k线在耀斑观测中的演变过程,从出现蓝翼增强到产生较大红移再到红移减小出现非线心反转谱线的过程。这个蓝翼增强来自于上流较冷($\sim 4\times10^4$ K)物质造成的蓝翼不透明度增强,支持了\textcites{Tei2018}观测到类似Mg \textsc{ii} h和k线的蓝翼增强的解释。
	\item 我们的模拟不能同时再现其他色球谱线如C \textsc{ii}和Si \textsc{iv}同时的演化过程,主要的问题可能来自于RADYN本身对冷却过程的处理不够完善以及我们没有在RH中使用同时包含Si \textsc{i}-\textsc{v}能级的Si原子模型。
	\item 在人工添加一个z方向的磁场后,我们利用RH模拟了He \textsc{iv} 10830 \mbox{\AA}谱线的Stokes参数在5F11模型中的演化过程。作为形成在高色球的谱线,He \textsc{i} 10830同样出现了较大红移的谱线轮廓。其Stokes参数$V/I$相比C级耀斑观测值稍大。
	\item 我们对不同非热电子束加热的M矮星耀斑模拟中的Mg \textsc{ii}谱线轮廓与形成高度进行了较为完备的分析。我们发现能够再现白光连续谱观测的非热电子束参数1F13$E_c85$加热的模拟大气并不能较好地再现Mg \textsc{ii}谱线轮廓。反而是另外两个1F11$E_c85$和1F13$E_c350$分别产生了类似于\textcites{Hawley2007}中观测到的蓝翼增强和红翼增强的谱线。
	\item 我们发现在M矮星中Mg \textsc{ii}三重线的线心形成高度在宁静大气中就已经和Mg \textsc{ii} h和k线接近,但是Mg \textsc{ii}三重线的线翼形成在更深的位置,它们的线翼辐射都对高截止能量$E_c$的电子束加热非常敏感。
	\item 我们利用手动在VALC大气中插入局部温度上升研究了不同高度不同程度加热对Mg \textsc{ii}和H$\alpha$谱线的影响。我们发现Ellerman炸弹光谱中出现的Mg \textsc{ii}线线翼增强可能需要在0.25-1.0 Mm的范围内有局部温度升高至接近$10^4$ K。
	\item 我们在低高度小尺度加热的模拟中发现了与\textcites{Hong2017b}类似存在低层大气温度升高时H$\alpha$谱线线心线翼强度减弱的情况。
\end{enumerate}


% vim:ts=4:sw=4
