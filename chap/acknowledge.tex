% Copyright (c) 2014,2016 Casper Ti. Vector
% Public domain.

\chapter{致谢}
我首先要感谢我的本科生科研导师田晖研究员在太阳物理研究领域提供的指导和帮助,同时提供了我与其他研究人员交流讨论学习的机会。特别是在IRIS卫星数据处理,谱线在不同情况下的形态特征等技能知识方面的指导。

本论文的部分工作是作者在2018年7-8月间于美国科罗拉多大学博尔德分校(\textit{University of Colorado Boulder})的国家太阳天文台(\textit{National Solar Observatory})访问交流时完成的。我非常感谢美国科罗拉多大学博尔德分校和国家太阳天文台的助理教授Adam. F. Kowalski与美国国家太阳天文台的天文学家Han Uitenbroek博士对我在辐射流体力学模拟,RADYN,RH代码使用等一系列问题上的帮助。我也要感谢北京大学本科生科研项目中的国家创新训练项目为我提供的资助。

作者还要感谢挪威奥斯陆大学(\textit{University of Oslo})的Mats Carlsson教授与作者在RADYN模拟及谱线致宽上的讨论,中科院紫金山天文台李瑛研究员和南京大学天文与空间科学学院洪杰博士对作者在辐射流体力学模拟方面的帮助以及美国NASA Goddard航天中心的Graham Kerr博士与作者在Si \textsc{iv}谱线模拟方面建设性的讨论。作者还要感谢ISSI/ISSI-BJ国际小组讨论会的全体参会人员,包括NASA Goddard航天中心Joel C. Allred博士,瑞士西北应用科学大学(\textit{University of Applied Sciences and Arts Northwestern Switzerland})的Lucia Kleint博士,捷克科学院(\textit{The Czech Academy of Sciences})的Petr Heinzel博士和日本京都大学(\textit{Kyoto University})的Akiko Tei博士等。

同时我要感谢北京大学科维理天文与天体物理研究所的Gregory J. Herczeg教授对我在恒星结构与演化知识学习的帮助和北京大学天文系的刘富坤教授开设的《理论天体物理》课程对我在辐射转移相关知识学习的帮助。

此外,我感谢田晖教授研究组中的Tanmoy Samanta,宋永亮,陈亚杰,张婧雯,谭光钰,朱瑞学长学姐,陈雨豪同学,高宇航学弟对我在本科生科研以及毕业论文中的帮助,特别是陈亚杰学长提供了紫外爆发和Ellerman炸弹的一系列光谱数据。我还要感谢我的室友赵辰兴宇,李京寰,刘立超,以及北京大学地球与空间科学学院2015级本科4班全体同学对我学习和生活上的帮助。

特别地,我还要感谢杨子浩同学在这两年本科生科研过程中给予我的帮助、支持、鼓励和启迪。他踏实认真严谨好学勤奋的研究态度一直是我学习的榜样。同时非常感谢他通读了我的论文,在文字修改上提出了很多重要的意见。

我感谢Github上的开发者Casper Ti. Vector(\href{mailto:CasperVector@gmail.com}{CasperVector@gmail.com})提供的\LaTeX 模板pkuthss对我论文排版提供的帮助。

最后我要感谢我的家人,尤其是我的母亲和奶奶对我的关心信任帮助对我一直以来恶劣性格和抑郁情绪的无限包容。没有她们我可能活不到今天。

%我还要在这里特地感谢我的女朋友王怡媛这一年多来在我情绪低落和陷入困境时给我的支持,帮助我一直走到现在。
% vim:ts=4:sw=4
